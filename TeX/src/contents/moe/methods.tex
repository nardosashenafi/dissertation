\section{Mixture of Expert Controllers}
\label{sec:moe_methods}

In this section, we present a framework that infers a mixture of expert
controllers for a hybrid dynamical system. 
%
This technique allows us to observe the effects of contacts in the closed loop
trajectories  and learn a switching mechanism to best control the hybrid system
in all modes.
%
We also learn optimal expert controllers that achieve a certain performance
objective through data-driven techniques.


%
Let $\gamma(x_0, u, T)$ denote a closed loop trajectory with initial state $x_0$
integrated for the time horizon $T$.
%
We pose the search over the parameters of the experts and the gating network as
\begin{equation}
    \begin{aligned}
        \underset{\psi, \theta}{\textrm{minimize}} 
        & & & \ell \Bigl(\gamma,u(\gamma; \psi, \theta) \Bigr)  , \\%
        \textrm{subject to}
        & & M(q) &\dd \dot{q} + h(x; \psi, \theta)\dd t - \dd R  = 0,\\%
        & & u = \{F_i&(x; \theta_i) \; | \; i  \sim \text{Categorical} (P(x; \psi)) \}
    \end{aligned}
    \label{eq:moe_opt}
\end{equation}
\noindent where $x = (q, \dot{q})$, and $\ell : \mathcal{X} \times \mathcal{U}
\rightarrow \mathbb{R}$ represents the running cost of closed loop trajectories
generated from~\eqref{eq:hybrid_dynamics}.
%
We follow Moreau's time stepping algorithm~\cite{glocker2005formulation} outlined
in Algorithm~\eqref{algo:moreau} to resolve the complementarity constraint
in~\eqref{eq:complementarity} and numerically integrate the measure differential
equations~\eqref{eq:hybrid_dynamics}.
%
At each integration step, the control law $u$ first samples a bin number from
$P(x;\psi)$; then it evaluates the expert controller $F_i$ that corresponds to
the sampled bin.
%
The running cost $\ell(\gamma, u)$ quantifies the performance of the trajectory
$\gamma$ under the current parameters $(\psi, \theta)$.
%
We use gradient-based techniques to adjust the decision parameters $(\psi, \theta)$
based on the running cost.

\begin{algorithm}
  \setstretch{1.2}
    \caption{Moreau's Time Stepping Algorithm}
    \label{algo:moreau}
    \small
    \hspace*{\algorithmicindent} \textbf{Input}: $\psi, \theta, x(0) = (q(0), \dot{q}(0))$
    \begin{algorithmic}[1]
      \State $\gamma \leftarrow  [x(0)]$ \Comment{Initial States}
        % \algrenewcommand\algorithmicindent{0em} % No indent
          \For{$t \in 0:\Delta t:T$} 
            \State $k \sim \text{Categorical}(P(x(t); \psi))$ \Comment{Sample a bin number}
            \State $u = F_i(x(t), \theta_k)$      \Comment{Expert in bin $k$}
            \State $t_M = t + \nicefrac{\Delta t}{2}$
            \State $q(t_M) = q(t) +  (\nicefrac{\Delta t}{2}) \dot{q}(t) $
            % \State $\mathcal{H} = \{j \;| \; g_{Nj}(q(t_M)) \leq 0, \; 1 \leq j \leq k\}$
            \State $\lambda_N, \lambda_T \leftarrow \text{Lemke}(q(t_M), \dot{q}(t))$\Comment{Lemke~\cite{acary2008numerical}} 
            \State $\dot{q}(t+\Delta t) = M^{-1}(W_T \lambda_T + W_N \lambda_N + h\Delta t) + \dot{q}(t)$
            \State $\gamma \leftarrow \gamma \cup [x(t+\Delta t)]$
          \EndFor
        \State \textbf{return} $\gamma$
    \end{algorithmic}
\end{algorithm}


\subsubsection{Performance Objective}
%
We present two viable choices for the running cost function. 
\begin{enumerate}
    \item Accumulated loss: is the total quadratic loss between the desired
    state $x^*$ and the state generated under the current control law. We also
    incur a cost on the control authority as follows.
    \begin{equation}
        \begin{gathered}
            \ell(x, u) = \frac{1}{2}(x - x^*)^\top Q (x - x^*) + \frac{1}{2} u^\top R u , \\
            \ell(\gamma, u) = \int_0^{T}  \ell(x(t), u)\dd t,
        \end{gathered}
    \label{eq:accumulatedLoss}
    \end{equation}
    \noindent where $Q$ is positive definite matrix and $R$ is a positive semi-definite matrix.
    %
    The corresponding likelihood is given by 
    \begin{align}
        \mathcal{L}(\gamma) = \int_{0}^{T} \Biggl( \sum_{i=1}^{N_F} \Bigl[ \ell \Bigl(x(t), F_i \Bigr) P\Bigl(c_i = i | x(t), \psi \Bigr) \Bigr] \Biggr) \dd t.
        \label{eq:accumulated_likelihood}
    \end{align}
    \item Minimum trajectory loss (MTL): some dynamical need to deviate further
    away from the desired equilibrium momentarily before converging to it.  
    %
    For instance, suppose we want to swing-up the simple pendulum to the upright
    equilibrium. 
    %
    For an underactuated pendulum, the controller needs to pump energy slowly,
    by moving closer and further away from the upright.
    %
    Accumulated loss incurs a lot of cost in such scenarios and the control
    search would get stuck in local minima.
    %
    In such cases, a successful loss function encourages trajectories that
    \it{eventually} \normalfont lead to a minimum cost.
    %
    Hence, we compose MTL as
    \begin{equation}
        \begin{gathered}
            t_{min} = \text{argmin}_{t} \; \ell(x(t), u)  \\
            \mathcal{L}(\gamma) = - \frac{\ell(x(t_{min}), u)}{C} \sum_{t=0}^{t_{min}} \log(P(c_i=i| x(t), \psi)) 
        \end{gathered} 
    \end{equation}
    \noindent where $C > 0$ is a normalization factor.
    %
    MTL allows us to train on a section of the trajectory that leads to the
    lowest cost. 
    %
\end{enumerate}

\subsubsection{State Sampling}

We intend to find a solution to the optimization problem
in~\eqref{eq:moe_opt} for all initial states $x_0$ in the
state space.
%
To efficiently sample the initial states, we use a combination of greedy and
explorative state sampling techniques.
%
The greedy state sampling, commonly known as \textsc{DAgger}~\cite{ross2011no},
generates trajectories using the current parameters and collects $N_d$ samples
from the visited states. 
%
This allows the training to refine the controller on the states it visits the
most.
%
We recover from locally optimal solutions through explorative state sampling,
which takes $N_r$ number of random samples from the state space.
%
In a single batch training, we compute the running cost as an expectation
over $N_{\mathcal{D}} = N_d+N_r$ samples as follows:
\begin{align*}
    J(\theta) = \mathbb{E}_{x_0 \sim \mathcal{D}_N}[ \ell(\gamma(x_0, u, T), u)]
\end{align*}
\noindent where $\mathcal{D}_N$ is a collection of $N_{\mathcal{D}}$ initial state samples.

We solve the optimization problem given in~\eqref{eq:moe_opt} through stochastic
gradient descent.
%
The full training procedure is outlined in Algorithm~\eqref{algo:moe_training}.
%
We leverage auto-differentiation techniques to compute the gradient of the
likelihood with respect to the learned parameters.
\begin{algorithm}
    \setstretch{1.2}
      \caption{Solution to the Optimization Problem~\eqref{eq:moe_opt}}
      \label{algo:moe_training}
      \small
      \begin{algorithmic}[1]
          \algrenewcommand\algorithmicindent{0em} % No indent
          \State $\mathcal{D}_N \gets \{x_0\}_{(N_{\mathcal{D}})}$  \Comment{$N_{\mathcal{D}}$ initial state samples} 
          \algrenewcommand\algorithmicindent{1.1em} % Change indent back to default
          \While{$i < $ \texttt{maximum iteration}}
          \State $J \gets 0$\Comment{Batch loss}
          \For{$z_0 \in \mathcal {D}_N$}
              \State $\gamma$ = Moreau($\phi, \theta, x_0$) \Comment Algorithm~\eqref{algo:moreau}
              \State $J \gets J + \mathcal{L}(\gamma)/N_{\mathcal{D}}$ \Comment{Batch loss}
          \EndFor
          \State $\theta \gets \theta + \alpha_i \nicefrac{\partial J}{\partial \theta}$\Comment{SGD step}
          \State $\mathcal{D}_N \gets \{x_0\}_{(N_{\mathcal{D}})}$\Comment{New initial state samples}
          \State $i \;\:\gets i + 1$
          \EndWhile
          \State \textbf{return} $\theta$
      \end{algorithmic}
  \end{algorithm}

