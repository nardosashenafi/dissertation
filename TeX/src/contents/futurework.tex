We tackle two main difficulties in the control of contact-rich robotic systems.
%
First, we address the computational complexities in designing multi-modal
controllers for hybrid dynamical systems.
%
In our framework, we use data-driven techniques to infer a mixture of expert
controller that switches between several policies.
%
The learning architecture also provides a gating network, which governs the
control switching scheme based on the observed states.
%
We use the linear complementarity formulation to accurately model contact-rich
systems and to train contact-aware MoE controller in simulation.
%
This data-driven technique finds controllers that react to the positive or
negative effects of contacts and impacts.
%
We demonstrate this behavior on the swing-up problem of the cartpole system.
%
We modify the standard cartpole problem to a multi-modal contact-rich mechanism
by introducing wall barriers on each sides of the system.
%
Using the MoE architecture, we successfully swing up the cartpole in simulation
and hardware.
%
The case study shows that the contact-aware controller finds a way to leverage
the impact from the wall barriers to catch the pendulum post-impact.
%

The second complication in the control of contact-rich systems involves
designing robust controllers under uncertain environmental conditions.
%
In particular, we raise the issue of operating a walking machine on uneven
terrain, where the exact elevation of the runway is unknown.
%
We characterize and tackle the effects of model uncertainties in the control of
dynamical system.
%
We leverage the robustness properties of Bayesian learning to infer stochastic
controllers that can achieve the desired performance under system parameter and
measurement uncertainties.
%
This technique is first demonstrated on smooth dynamical systems, such as the
simple pendulum and the inertia wheel pendulum, in simulation and hardware.
%
Then, we extend the Bayesian framework to contact-rich systems, such as the
rimless wheel walking machine, and demonstrate the robustness properties of the
Bayesian technique against a deterministic framework.

The control design techniques presented in this work have laid the foundation
for numerous future research directions.
%
The MoE architecture can be applied to address many interesting challenges in
the control of contact-rich robots.
%
For instance, we can utilize the MoE framework to select an optimal way to
manipulate and maneuver an object in space.
%
This framework can be used to select one of many object manipulation techniques,
such as pushing, lifting or sliding an object, based on the obstacles present in
the environment.
%
We also intend to combine the multi-modal nature of the MoE framework with the
robustness properties of Bayesian inference. 
%
This is most useful when the robotic system interacts with a dynamic
environment.
%
An example of such scenario can be found in the cartpole system with wall
barriers.
%
If these wall barriers are not static, the controller needs to reason about the
uncertainty in the walls' positions, and learn a controller robust against the
unexpected impacts from these barriers.
